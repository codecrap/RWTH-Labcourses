% !TeX encoding = UTF-8
% !TeX spellcheck = en_US
% !BIB TS-program = biber
% basic packages and document settings
\documentclass[a4paper,12pt]{article}
\usepackage[english]{babel}
\usepackage[utf8]{inputenc}
\usepackage[T1]{fontenc}
\usepackage[a4paper]{geometry}
\geometry{top = 30mm, bottom = 25mm, left = 25mm, right = 25mm}
\usepackage{setspace}
\onehalfspacing
\raggedbottom
\pdfcompresslevel9

% mathmode-related packages
\usepackage{mathtools}
\usepackage{physics}
\usepackage{amsmath}
\usepackage{amsthm}
\usepackage{amsbsy}
\usepackage{mathrsfs}
\usepackage{amssymb}
\usepackage{amstext}
\usepackage{amsfonts}
\usepackage{tikz}
\usepackage{siunitx}
%\usepackage{IEEEtrantools}

% misc
\usepackage{esdiff}
\usepackage{multirow}
\usepackage{blindtext}
\usepackage{todonotes}
\usepackage{abstract}
\usepackage{appendix}
\usepackage[bottom]{footmisc}
\usepackage{listings}

% packages requiring setup arguments


% graphics and floats
\usepackage{graphicx}
\graphicspath{{../Figures/}}
\usepackage{float}
\usepackage{wrapfig}
%\usepackage{subfloat}
\usepackage{subcaption}
%\usepackage{caption}
\usepackage[rightcaption]{sidecap}
\usepackage{tabularx}
\usepackage{adjustbox}
%\usepackage{svg}
\usepackage{epstopdf}
\usepackage{grffile}	% handle file names with dots, spaces etc.
%\usepackage{flafter}
\usepackage{rotating}
%\usepackage{floatrow}
%\floatsetup[figure]{capposition=beside,capbesideposition={top,right}}

%\usepackage{epstopdf}
%\epstopdfDeclareGraphicsRule{.pdf}{png}{.png}{convert #1 \OutputFile}
%\AppendGraphicsExtensions{.pdf}

\usepackage{xcolor}
%\definecolor{rwth-dark}{HTML}{176daf}
\definecolor{rwth-dark}{RGB}{0,84,159}
%\definecolor{rwth-light}{HTML}{8abae3}
\definecolor{rwth-light}{RGB}{142,186,229}

%/**
%* Generated by Gpick 0.2.5
%* RWTH Dark: #176daf, rgb(23, 109, 175), hsl(32, 43%, 69%)
%* RWTH Light: #8abae3, rgb(138, 186, 227), hsl(195, 73%, 89%)
%*/

\usepackage{hyperref}
\hypersetup{hidelinks=true,
colorlinks=true,
allcolors=rwth-dark}
\usepackage[nameinlink,capitalise]{cleveref}
%\usepackage[hyphens]{url}
%\urlstyle{sf}
%\usepackage{breakurl}


% header & footer settings
\usepackage{fancyhdr}
\pagestyle{fancy}
%\renewcommand{\chaptermark}[1]{\markboth{#1}{}} % with this we ensure that the chapter and section headings are in lowercase.
\renewcommand{\sectionmark}[1]{\markright{\thesection\ #1}}
\fancyhf{} % delete current header and footer

%\fancyhead[LE,RO]{\large\thepage}
\fancyhead[L]{\large\rightmark\thepage}
\fancyhead[R]{\large\leftmark}

\renewcommand{\headrulewidth}{0.3pt}
\renewcommand{\footrulewidth}{0pt}
%\addtolength{\headheight}{0.5pt} % space for the rule

\fancyfoot[C]{\thepage}

\fancypagestyle{plain}
{
	\fancyhead{} % get rid of headers on plain pages
	\renewcommand{\headrulewidth}{0pt} % and the line
}

% ========== command definitions =================================
\newcommand{\Thickline}{\rule{\linewidth}{0.4mm}}
\newcommand{\Thinline}{\rule{\linewidth}{0.1mm}}

\newcommand{\id}{\, \mathrm{d}}

\definecolor{codegray}{gray}{0.9}
\newcommand{\code}[1]{\colorbox{codegray}{\texttt{#1}}}

\newcommand{\skippage}{\clearpage{\thispagestyle{empty}\cleardoublepage}}

\def\@esphack{%
	\relax
	\ifhmode
	\spacefactor\@savsf
	\ifdim\@savsk>\z@
	\ignorespaces
	\fi
	\fi}

\title{\LARGE title}
\date{}


\begin{document}
	
\begin{titlepage}
	\thispagestyle{empty}
	\newgeometry{top=20mm, left=20mm, right=20mm, bottom=20mm}
	
%	\begin{minipage}{0.35\textwidth}
%		\begin{flushleft}
%
%
%		\end{flushleft}
%	\end{minipage}
%	\hfill
%	\begin{minipage}{0.65\textwidth}
%		\begin{flushright}
%
%		\end{flushright}
%	\end{minipage}
	
	\vspace{3cm}
	\begin{center}
		\Thickline
		\vskip -0.45cm
		\Thinline
		\vspace{0.5cm}
		
		\huge{ \textbf{ Report T2 - Gamma spectroscopy and Compton scattering } } 
		
		\Thinline
		\vskip -1cm
		\Thickline 
		\vspace{1cm}

		
	\end{center}
	\vfill
	
\end{titlepage}
	
	
	
\skippage
\pagenumbering{roman}
\thispagestyle{plain}

\tableofcontents
%	\newpage
\listoffigures

\begingroup
\let\cleardoublepage\relax
\let\clearpage\relax
\listoftables
\endgroup
%\listoftables

\skippage

\pagenumbering{arabic}
\setcounter{page}{1}
\restoregeometry
\thispagestyle{fancy}


\skippage
	
\section{Introduction}
In this experiment we want to analyse the Energy of photons emitted by radioactive probes with a szintillaor detector. First we will use Materials with low radiation and with well known energy peaks to calibrate the detector. Then we will use that to analyse the spectrum of a source with stronger radiation and establish a link between scattering angle and the energy of compton scatterd photons. 

\newpage

\section{Gamma spectroscopy}

\subsection{Theory}
We are looking at photons with ernegies from 5keV up to 2Mev. They are three relevant interactions of photons with matter within this energy range.\\
Photoelectric effect ($E_\gamma \sim E_B$):\\
Incoming photon with energy Eg is absorbed by an electron with binding ernergy $E_B$. That elctron leaves the Atom with kinetic Energy $E_{\mathrm{kin}} = E_\gamma - E_B$. X-ray radiation follows because of the empty position beeing filled b an electron of a higher shell.\\
Compton scattering ($E_\gamma \gg E_B$):\\
Incoming Photon is scattered at an electron. It is not absorbed but transmits enrgy to the electron. The maximum transmited energy is given by $E_C$. That maximum ist obtained through frontal collision (stattering angle $\theta = 180^\circ$).\\
Pair production ($E_\gamma \geqslant 2m_ec^2(1+\frac{m_e}{M})$):\\
If the Energy is greater than the mass of two electrons, the photon can decay in a positron and an electron. Given an additional interaction partner with mass $M$ for momentum conservation, a (non-virtual??) positron-electron-pair can be produced. This pair then decays into two photons.\\
\\
The expected energy peaks for an incoming photon with energy $E_\gamma$ are the following:\\
Photo peak at $E_\gamma \leftarrow$ All energy is absorbed.\\
Compton edge at $E_C \leftarrow$ Compton collision with frontal collision and undetected scattered photon.\\
Escape peak at $E_\mathrm{esc}^{(1)} = E_\gamma - m_ec^2 \leftarrow$ Pair production with one undetected final state photon.\\
Double escape peak $E_\mathrm{esc}^{(2)} = E_\gamma - 2m_ec^2 \leftarrow$ Pair production with two undetected final state photon.\\
Backscatter peak $E_R = E_\gamma - E_C \leftarrow$ Compton effect outside of the scintillator with absorbtion of the scattered photon.\\

\subsection{Setup}
A radioactive probe is placed in front of the scintillator at distance $r_0$. 

\newpage

\section{Compton scattering}

\subsection{Theory}
Energy of scattered photons: $E_\gamma^\prime = E_\gamma \cdot \frac{1}{1+a(1-\cos\theta)}$\\
count rate: $m = \frac{A \cdot I_\gamma}{4 \pi r_0^2} \cdot \eta \cdot \epsilon \cdot N_e \cdot \frac{d\sigma}{d\Omega} \cdot \frac{F_D}{r^2}$

\subsection{Setup}

\subsubsection{conventional geometry}
The source is placed on the edge of a rottary table, allowing different angles. In the middle of the table we place the scattering body (either a steel or an aluminium cylinder). To shield the radiation coming directly from the source from the detector, we use convenientl shaped lead coils.\\
Distance from source to scattering body: $r_0 = \SI{49+-1}{mm}$\\
Distance from scattering bod to detector: $r = \SI{272+-1}{mm}$\\

\subsubsection{ring geometry}
The source is aligned with the detector, but shielded of it by a lead clinder in the middle. The scattering body is an aluminium ring and the whole experiment is axially symetric.



\end{document}