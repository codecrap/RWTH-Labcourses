\documentclass[12pt]{beamer}
\usepackage[utf8]{inputenc}
\usepackage[ngerman]{babel}
\usepackage{graphicx}
\usepackage{wasysym}
\usetheme{Singapore}
\usepackage{amsmath}

\setbeamercovered{transparent}
\setbeamertemplate{navigation symbols}{}
\setbeamertemplate{footline}[frame number]
\setbeamertemplate{caption}[numbered]

\title{\emph{Akustik}}
\author{\underline{Gruppe 2}  \\  Adelind Elshani \\ \textbf{Olexiy Fedorets} \\ Bilal Malik \\ Tobias Wild}
\date{\today}	

\begin{document}
	
\begin{frame}[plain]
\titlepage
\end{frame}

\begin{frame}{Versuchsziele}
\begin{itemize}
\item Bestimmung der Schallgeschwindigkeit in Luft
\item Bestimmung der Elastizitätsmodule von Metallen
\item Schwebung bei einer leicht verstimmten Gitarre 
\item Bestimmung der Saitenspannung einer Gitarrensaite 
\item Analyse des Obertonspektrums einer Saite
\end{itemize}
\end{frame}

\begin{frame}{Gliederung}
\begin{enumerate}
\item {Bestimmung der Schallgeschwindigkeit in Luft}
	\begin{enumerate}
	\item {Grundlagen}
	\item {Bestimmung über die Laufzeit}
	\item {Bestimmung über Resonanzfrequenzen}
	\item {Bestimmung über Druckbäuche}
	\item {Fazit}
	\end{enumerate}
\item {Schallgeschwindigkeit in Festkörpern}
	\begin{enumerate}
	\item {Grundlagen}
	\item {Bestimmung der Elastizitätsmodule}
	\item {Fazit}
	\end{enumerate}
\item {Physik der Gitarre}
	\begin{enumerate}
	\item {Grundlagen}
	\item {Schwebung}
	\item {Saitenspannung}
	\item {Frequenzspektrum}
	\item {Fazit}
	\end{enumerate}
\end{enumerate}
\end{frame}

\begin{frame}{Schallgeschwindigkeit in Luft - Grundlagen}
\item {wesentliche Eigenschaften einer Schallwelle sind Schallschnelle und Schalldruck}
\item {bei idealem Gas hängt Schallgeschwindigkeit nur von Temperatur ab}
\item {$v = \sqrt{\frac{p_0 \cdot \kappa}{\rho} = v_0 \cdot \sqrt{\frac{T}{T_0}}$ \\
	   mit $v_0 = \sqrt{\frac{R \cdot \kappa \cdot T_0}{M_{mol}}}$} 
\item {bei schallfestem Ende bilden sich bei bestimmten Frequenzen stehenden Wellen aus}
\item {$f_n = n \cdot \frac{v}{2L}$ bzs $L_n = n \cdot \frac{v}{2f}$}
\item {Lage der Druckknoten: $x_n = n \cdot \frac{\lambda}{2} - \frac{\lambda}{4}$}
\end{frame}


\end{document}