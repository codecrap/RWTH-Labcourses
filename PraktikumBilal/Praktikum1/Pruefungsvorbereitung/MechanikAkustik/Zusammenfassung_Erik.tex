\documentclass[10pt,a4paper]{article}
\usepackage[utf8]{inputenc}
\usepackage[german]{babel}
\usepackage[T1]{fontenc}
\usepackage{amsmath}
\usepackage{amsfonts}
\usepackage{amssymb}
\usepackage{makeidx}
\usepackage{graphicx}
\usepackage[left=2cm,right=2cm,top=2cm,bottom=2cm]{geometry}
\author{Erik Zimmermann}
\usepackage{float}
\title{Praktikum Formelblatt}
\begin{document}
\section{Mechanik}
\subsection{Einzelpendel}
\subsubsection{Formeln}
\begin{equation}
\omega^2=\frac{D_P}{J_P}=\frac{m g\cdot l_P}{0.5\cdot m_p r_P^2+m_p l_p^2}
\end{equation}
\begin{equation}
\Rightarrow g= \omega^2 l_P\cdot(1+\frac{1}{2}\frac{r_P^2}{l_P^2})
\end{equation}
\begin{equation}
\omega=2\pi f=\frac{2\pi}{T}\text{ mit } T=\frac{t_e-t_a}{n}
\end{equation}
\subsubsection{Parameter Cassy}
\begin{table}[H]\centering
\begin{tabular}{|c|c|}
\hline 
Intervall & 10 ms \\ 
\hline 
Anzahl & 16000 \\ 
\hline 
\end{tabular} 
\end{table}
\subsubsection{Sonstiges}
\begin{align}
l\approx 0.68m,\hspace{0.5cm}\sigma_l=0.5..1cm\\
\omega_S\approx \omega_P~(0.5\text{ Prozent}) \rightarrow 1-\frac{\omega_S}{\omega_P}<0.005
\end{align}
\subsection{Doppelpendel}
\subsubsection{Formeln}
\begin{align}
\kappa=\frac{\omega_{sf}^2-\omega_s^2}{\omega_{sf}^2+\omega_s^2}=\frac{f_+^2-f_-^2}{f_+^2+f_-^2}\\
\frac{1}{\kappa}=1+\frac{m g l_P}{D_F}\cdot \frac{1}{l_F^2}
\end{align}
\section{Akustik Lars und Erik}
\subsection{Schallgeschwindigkeit in Festkörpern}
\subsubsection{Formeln}
\begin{align}
v=\sqrt{\frac{E}{\rho}}\hspace{1cm} \lambda= 2\cdot L\\
\Rightarrow E=\rho\cdot f_0^2 \cdot 4L^2\\
\text{mit } \rho= \frac{M}{V}=\frac{4M}{L\pi D^2}
\end{align}
\subsubsection{Parameter Cassy}
\begin{table}[H]\centering
\begin{tabular}{|c|c|}
\hline 
Intervall & 100 $\mu$s \\ 
\hline 
Anzahl & 16000 \\ 
\hline 
\end{tabular} 
\end{table}
\subsubsection{Sonstiges}
\begin{itemize}
\item Mikro auf minimale Empfindlichkeit im Amplituenmodus
\item Mikro schaltet sich nach 10 min aus(!)
\item Stab darf nur an einem einzigen Punkt befestigt sein
\end{itemize}
\subsection{Schwebung der Gitarre}
\subsubsection{Formeln}
\begin{align}
f_k=\frac{f_++f_-}{2} \hspace{1cm} f_{sch}=\frac{f_+-f_-}{2}
\end{align}
\subsubsection{Parameter Cassy}
\begin{table}[H]\centering
\begin{tabular}{|c|c|}
\hline 
Intervall & 500 $\mu$s \\ 
\hline 
Anzahl & 10000 \\ 
\hline 
Trigger & z.B. 0.3V \\ 
\hline 
\end{tabular} 
\end{table}
\subsubsection{Sonstiges}
\begin{itemize}
\item aus FFT: $f_+$ und $f_-$
\item ablesen bzw. abzählen: $f_{k}$ und $f_{sch}$
\item aufpassen ob nach f oder $\omega$ gefragt ist
\item Mikro ca. 50 cm über die Gitarre
\end{itemize}
\subsection{Materialeigenschaften einer Saite}
\subsubsection{Formeln}
\begin{equation}
f= \frac{1}{2}\sqrt{\frac{T}{\mu}}\cdot \frac{1}{l}
\end{equation}
\subsubsection{Sonstiges}
\begin{itemize}
\item Frequenz gegen die Länge der Saite (ggf. an mehreren Bünden)
\item Bünde richtig ablesen! :)
\end{itemize}
\end{document}