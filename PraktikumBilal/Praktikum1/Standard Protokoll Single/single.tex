\documentclass[12pt,a4paper]{article}
\usepackage[utf8]{inputenc}
\usepackage[german]{babel}
\usepackage[T1]{fontenc}
\usepackage{amsmath}
\usepackage{amsfonts}
\usepackage{amssymb}
\usepackage{graphicx}
\usepackage[left=2.5cm,right=2.5cm,top=2cm,bottom=2cm]{geometry}
\usepackage{float}
\author{Gruppe C14 \\ Julián Häck, Martin Koytek, Lars Wenning, Erik Zimmermann}
\begin{document}
\section{z.B. Widerstand, Teilversuch 4.1}
\subsection{Versuchsbeschreibung}
Kurze Darstellung der physikalischen Grundlagen und Ziele der Versuche, die zum Verständnis
des Versuches/Protokolls benötigt werden. (max. 1 Seite)
\subsection{Versuchsaufbau und Durchführung}
Genaue Beschreibung der verwendeten Aufbauten unter Verwendung von Skizzen oder Photos
Beschreibung der Messwerterfassungseinstellungen (eingestellte Messzeiten, Messbedingungen,
Trigger, Anzahl der Messungen) und der Durchführung der Versuche. (max. 1 Seite)
\subsection{Versuchsauswertung}

\subsubsection{Rohdaten}
1 Seite
\subsubsection{Transformation der Rohdaten}
Transformation der Rohdaten und Modellanpassung. (1 Seite)
\subsubsection{Analyse}
Analyse der Daten inklusive Fehlerrechnung Residuen und Pullverteilung. (1 Seite)
\subsubsection{Fazit}
Diskussion der Ergebnisse und Vergleich der erzielten Ergebnisse mit theoretischen Vorhersagen.
(1 Seite)

\end{document}