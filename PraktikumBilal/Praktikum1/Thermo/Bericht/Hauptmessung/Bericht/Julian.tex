\documentclass[10pt,a4paper]{article}
\usepackage[utf8]{inputenc}
\usepackage[german]{babel}
\usepackage[T1]{fontenc}
\usepackage{amsmath}
\usepackage{amsfonts}
\usepackage{amssymb}
\usepackage{makeidx}
\usepackage{graphicx}
\usepackage[left=2cm,right=2cm,top=2cm,bottom=2cm]{geometry}
\usepackage{float}
\author{Erik Zimmermann}
\begin{document}

\section{Bestimmung der Verdampfungsenthalpie von Wasser}
\subsection{Versuchsbeschreibung}
%Kurze Darstellung der physikalischen Grundlagen und Ziele der Versuche, %die zum Verständnis
%des Versuches/Protokolls benötigt werden. (max. 1 Seite)
Zur Bestimmung der Verdampfungsenthalpie wird die Verdampfungswärme in einem isochoren Prozess bestimmt, wodurch die Volumenarbeit verschwindet. Somit ist die Verdampfungsentalpie gleich der Verdampfungswärme. Grundlegend für den Versuch ist die Clausius-Clapeyonschen Gleichung:
\begin{equation}
\frac{dp}{dT}=\frac{\nu \Lambda}{T(V_1-V_2)}
\end{equation}
mit der Stoffmenge $\nu$, der Verdampfungswärme $\Lambda$ und der Differenz der Volumen(Gas,Flüssigkeit).
Unter der Annahme, das das Gasvolumen von Wasserdampf deutlich größer (Faktor 1200) ist als das Volumen von Wasser (flüssig), ergibt sich die DGL zu
\begin{equation*}
\frac{dp}{dT}=\frac{\nu \Lambda}{T\cdot V_{gas}}
\end{equation*}
Mit der Näherung des idealen Gases ($p\cdot V=\nu R T$) lässt sich die DGL lösen:
\begin{equation}
ln(\frac{p}{p_0})=-\frac{\Lambda}{R}(\frac{1}{T}-\frac{1}{T_0})
\end{equation}
bzw.
\begin{equation}
\ln(p)=-\frac{\Lambda}{R}\cdot \frac{1}{T}+c \text{ mit } c=const
\end{equation}
Nun wird der Druck und die Temperatur des Wasserdampfes beim Abkühlen gemessen und anschließend $\ln(p)$ gegen $\frac{1}{T}$ aufgetragen. Die Steigung ergibt sich dann zu $-\frac{\Lambda}{R}$ aus der dann die Verdampfungswärme $\Lambda$ bestimmt wird.


\end{document}
