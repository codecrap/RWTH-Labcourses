\documentclass[12pt,a4paper]{article}
\usepackage[utf8]{inputenc}
\usepackage[german]{babel}
\usepackage[T1]{fontenc}
\usepackage{amsmath}
\usepackage{amsfonts}
\usepackage{amssymb}
\usepackage{graphicx}
\usepackage[left=2.5cm,right=2.5cm,top=2cm,bottom=2cm]{geometry}
\usepackage{float}
\author{Gruppe C14 \\ Julián Häck, Martin Koytek, Lars Wenning, Erik Zimmermann}
\begin{document}
\section{z.B. Widerstand, Teilversuch 4.1}
\subsection{Versuchsbeschreibung}
%Kurze Darstellung der physikalischen Grundlagen und Ziele der Versuche, %die zum Verständnis
%des Versuches/Protokolls benötigt werden. (max. 1 Seite)
\begin{align*}
g=\omega^2 l_p (1+\frac{r_p^2}{2 l_p^2}) \\
\sigma_g=\sqrt{(2\omega l_p (1+\frac{r_p^2}{2 l_p^2}))^2 \cdot \sigma_{\omega}^2+(\omega^2 \cdot \frac{r_p}{l_p})^2 \cdot \sigma_r^2+(\omega^2(1-\frac{r_p^2}{2 l_p^2}))^2 \cdot \sigma_l^2} \\
\kappa =\frac{D_F l_F^2}{mgl_s + D_F l_F^2}= \frac{\omega_{sf}^2-\omega_{s}^2}{\omega_{sf}^2+\omega_s^2} \\
J \cdot \ddot{\phi} = -m_s \cdot g \cdot l \cdot \phi \\
T^2=\frac{1}{g} \cdot 4\pi^2 \cdot \frac{J}{m \cdot l_s} \\
J_p=\frac{1}{2} \cdot m_p r_p^2 + m_p l_p^2 \\
\frac{1}{\kappa}=1+\frac{ml_sg}{D_F}\cdot \frac{1}{l_F^2}
\end{align*}
\subsection{Versuchsaufbau und Durchführung}
Genaue Beschreibung der verwendeten Aufbauten unter Verwendung von Skizzen oder Photos
Beschreibung der Messwerterfassungseinstellungen (eingestellte Messzeiten, Messbedingungen,
Trigger, Anzahl der Messungen) und der Durchführung der Versuche. (max. 1 Seite)
\subsection{Versuchsauswertung}

\subsubsection{Rohdaten}
1 Seite
\subsubsection{Transformation der Rohdaten}
Transformation der Rohdaten und Modellanpassung. (1 Seite)
\subsubsection{Analyse}
Analyse der Daten inklusive Fehlerrechnung Residuen und Pullverteilung. (1 Seite)
\subsubsection{Fazit}
Diskussion der Ergebnisse und Vergleich der erzielten Ergebnisse mit theoretischen Vorhersagen.
(1 Seite)

\end{document}