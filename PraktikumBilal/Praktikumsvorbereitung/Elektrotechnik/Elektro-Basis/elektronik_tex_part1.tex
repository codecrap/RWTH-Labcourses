\author{Martin Koytek - Lars Wenning - Erik Zimmermann}
\title{E-Technik}
\documentclass[10pt,a4paper]{article}
\usepackage[utf8]{inputenc}
\usepackage[german]{babel}
\usepackage[T1]{fontenc}
\usepackage{amsmath}
\usepackage{amsfonts}
\usepackage{amssymb}
\usepackage{graphicx}
\begin{document}
\maketitle
\section{Vorbereitung}
Ohmsches Gesetz:  
\begin{equation}
U=R*I
\end{equation}
Kirchhoffsche Regeln: 
Knotenregel	
\begin{equation} \sum_{i}{I_i}=0 \end{equation} 
Maschenregel 
\begin{equation}\sum_{i}{U_i}=0\end{equation} 
Schaltung von Widerständen: 
Reihenschaltung
Addieren der Einzelwiderstände
\begin{equation}
U_g=\sum_{i}{U_i}=\sum_{i}{R_i*I}
\end{equation}
Parallelschaltung
Einzelwiderstände reziprok addieren  
\begin{align}
U_g=R_{ges}*I \\ 
\text{  mit  } R_{ges}=(\frac{1}{\sum_{i}{R_i}})^{-1}
\end{align}
Kondensatoren:
Als einen Kondensator bezeichnet man ein Bauteil in einem Schaltkreis mit entgegengesetzten Leiterflächen
unterschiedlicher Polung
Ladung: 
\begin{equation}
Q=C*U
\end{equation}
wobei C die Kapazität darstellt mit [C]= 1 Farad = 1 F
Plattenkondensator
$\Rightarrow$ zwischen Platten gibt es keine Ladung
$\Rightarrow$ Laplace Gleichung  
\begin{equation}
\frac{\partial^2 \phi}{\partial x^2}=0
\end{equation}
als Vereinfachung der Poisson Gleichung, wenn im Raumgebiet keine Ladung 
\begin{align}
\vartriangle \phi =\frac{-\rho}{\epsilon_0} \notag \\
\nabla \cdot \nabla \phi =\vartriangle \phi = 0 \text{ für } \rho = 0 \\
\phi = \int _P^\infty E ds
\end{align}
wobei P ein beliebiger Raumpunkt ist
\begin{align}
ds=d^2x&=dA \\
\delta(r)&=\text{Ladungsverteilung} \\
\Rightarrow \frac{\partial^2\phi}{\partial x^2}&=0\\
\Rightarrow \phi&=ax+b \\
U&=\phi_1-\phi_2 \\
\Rightarrow \phi_1&=\phi(0)\\
\phi_2\phi(d)&=a*d+\phi_1 \\ \Rightarrow a&=\frac{U}{d} \\
\Rightarrow\phi(x)&=-\frac{U}{d}*x+\phi_1 \\
E=-\nabla \phi&=\frac{U}{d}
\end{align}
Elektrische Feldstärke:
\begin{align}
E = \frac{U}{d} &= \frac{Q}{C} \cdot \frac{1}{d}\\
\Rightarrow C &= \frac{Q}{E d}\\
E_{empirisch} &= \frac{Q}{A \epsilon_0}\\
\Rightarrow \frac{U}{d} = \frac{Q}{A\cdot\epsilon_0} &= \frac{C\cdot U}{A \cdot \epsilon_0}
\end{align}
Kugelkondensator
\begin{align}
Q &= A_{Kugel} \cdot \sigma\text{ mit $\sigma$: Flächenladungsdichte}\\
&=4\pi R^2 \sigma \text{ mit } \vartriangle \phi = \frac{-\rho}{\epsilon_0}\\
\phi_{el} &= \int{E dA} =E \cdot 4 \pi r^2 = \frac{Q}{\epsilon_0}\\
\Rightarrow E &= \frac{Q}{4\pi\epsilon_0r^2} \vec{r}\\
\phi_{innen} &= \frac{Q}{4\pi\epsilon_0 a}
\end{align}
da Potential außerhalb mit 1/r abfällt. Im Inneren ändert sich das Potential nicht
\begin{align}
\phi_{zwischen} &= \frac{Q}{4\pi\epsilon_0 r}\\
\phi_{aussen} &= \frac{Q}{4\pi\epsilon_0 b}
U = \phi_{innen} - \phi_{aussen} &= \frac{Q}{4\pi\epsilon_0}\frac{b-a}{ab}\\
C=\frac{Q}{U}&= 4\pi\epsilon_0 a\\
Q &= 4\pi\epsilon_0 a U
\end{align}
Kondensatoren:
\begin{align}
\text{Parallel: } C &= \sum{C_i}\\
\text{Reihe: } \frac{1}{C} &= \sum{\frac{1}{C_i}}
\end{align}
Energie des Elektrischen Feldes:
\begin{align}
W &= \frac{1}{2} \frac{Q^2}{C} = C\cdot U^2\\
&= \int{(\phi_{aussen} - \phi_{\infty}) dQ}\\
&= \int{\phi_{aussen} dQ} \text{ , da }\phi_{\infty} \rightarrow 0\\
&= \int{\frac{Q}{4\pi\epsilon_0a} DQ}
&= \frac{1}{2} \frac{Q^2}{C}
\end{align}
\end{document}

