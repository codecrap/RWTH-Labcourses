\documentclass[10pt,a4paper]{article}
\usepackage[utf8]{inputenc}
\usepackage[german]{babel}
\usepackage[T1]{fontenc}
\usepackage{amsmath}
\usepackage{amsfonts}
\usepackage{amssymb}
\usepackage{makeidx}
\usepackage{graphicx}
\author{Erik Zimmermann}
\title{Auf-/ Endladung eines Kondensators}
\begin{document}
\maketitle
\newpage

\tableofcontents

\newpage
\section{Aufladung}
\begin{itemize}
\item <Bild Schaltung>
\item bei t=0: max Stromstärke:
\begin{equation}
I_0= \frac{U_0}{R}
\end{equation}
\item sobald Schalter wieder offen --> 2. Kirchhoff'sches Gesetz
\begin{equation}
U_0 - U_C=U_R= R*I(t)
\end{equation}
\item mit $I= \frac{dQ}{dt}$ , $Q=C*U$ --> $dQ=C*dU$
\item folgt $U_0-U_C(t)=R*C \frac{dU_c}{dt}$
\item --> DGL mit Randwert $Q(t=0)=0$ 
\begin{align}
U_0-U-C(t)&=R*C* \frac{dU_c}{dt}\notag \\
\Leftrightarrow (U_0-U_C)*dt&=R*C*dU_C \notag \\
\Leftrightarrow \int^{t}_{0}\limits {dt}&=R*C* \int^{U_C(t)}_{0}\limits {\frac{dU_C}{U_0-U_C} } \notag \\
\Leftrightarrow U_C(t)&=U_0*(1-e^(\frac{-t}{R*C}))\notag \\ 
\Rightarrow I(t)= \frac{U_0}{R}*e^(\frac{-t}{R*C})&=I_0*e^(\frac{-t}{\tau})
\end{align}
\end{itemize}

\section{Entladung}
\begin{itemize}
\item bei $t=0$ , Spannungsquelle durch Schalter überbrückt
\item --> Masche: $R*I(t)+U_C(t)=0$
\item mit $I= \frac{dQ}{dt}$, $Q=C*U$ folgt DGL:
\begin{align}
\int^{U_C}_{U_0}\limits {\frac{dU_C}{U_C}}=\frac{-1}{R*C} * \int^{t}_{0}\limits {dt } \notag \\
\Leftrightarrow U_C(t)= U_0* e^(\frac{-t}{R*C})
\Rightarrow I(t)=\frac{U_0}{R}*e^{(\frac{-t}{\tau})}
\end{align}
\end{itemize}
\end{document}
