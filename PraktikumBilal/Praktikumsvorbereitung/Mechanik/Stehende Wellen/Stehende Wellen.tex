\documentclass[10pt,a4paper]{article}
\usepackage[utf8]{inputenc}
\usepackage{amsmath}
\usepackage{amsfonts}
\usepackage{amssymb}
\usepackage{graphicx}
\begin{document}
\section{Stehende Wellen}
\begin{itemize}
\item Eine stehende Welle ist eine Welle mit der Gruppengeschwindigkeit = 0
\item Schwingungsbauch im Abstand d vom Mittelpunkt
\begin{equation}
d=n \cdot \frac{\lambda}{2}
\end{equation}
\item Schwingungsknoten
\begin{equation}
d=(n+\frac{1}{2}) \cdot \frac{\lambda}{2}
\end{equation}
\item Superposition zweier gegeneinander laufenden Wellen:
\begin{align}
x_1(\vec r,t) &=x_0 \cdot cos(\vec k \cdot \vec r -\omega t) \\
x_2(\vec r,t) &=x_0 \cdot cos(-\vec k \cdot \vec r -\omega t) \\
x(\vec r,t) &=-2 x_0 \cdot cos(\vec k \cdot \vec r) cos(\omega t)
\end{align}
\item Für $\vec r = 0$
\begin{equation}
x(\vec r,t)=2x_0 \cdot cos(-\omega  t)=0
\end{equation}
\item Ein festes Ende:
\begin{equation}
\lambda_n = \frac{4l}{2n+1}
\end{equation}
\item Zwei feste Enden:
\begin{equation}
\lambda_n = \frac{2l}{n+1}
\end{equation}
\item Kein festes Ende:
\begin{equation}
\lambda_n =\frac{2l}{n+1}
\end{equation}
\subsection{Resonanzprinzip}
\item Phasenverschiebung bei Reflexion und Amplitude $x_0=0$:
\begin{align}
k L &=n \pi \\
\lambda &=\frac{2L}{n}
\end{align}
\item Differenz der Frequenz zweier Moden:
\begin{equation}
\Delta \nu = \frac{c}{2L}
\end{equation}
\end{itemize}
\end{document}