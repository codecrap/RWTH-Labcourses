\documentclass[10pt,a4paper]{article}
\usepackage[utf8]{inputenc}
\usepackage[german]{babel}
\usepackage[T1]{fontenc}
\usepackage{amsmath}
\usepackage{amsfonts}
\usepackage{amssymb}
\usepackage{graphicx}
\begin{document}
\section{Verdampfungsenthalpie}
Die Verdampfungswärme $\Delta Q_v$ ist die Wärmemenge, die benötigt wird, um eine bestimmte Menge einer Flüssigkeit vom flüssigen in den gasförmigen Aggregatzustand zu bringen.
Beim umgekehrten Prozess wird genau diese Wärmemenge wieder als Kondensationswärme frei.
Für die Enthalpie (H) gilt:
\begin{align}
H &= U + p \cdot V \\
\Rightarrow \Delta H &= \Delta U + p \Delta V + \Delta p V \notag
\end{align}
Für isobare Prozesse gilt: 
\begin{equation}
\Delta Q = \Delta U + p \Delta V = \Delta H_v
\end{equation}
daher wird in diesem Fall von der Verdampfungsenthalpie gesprochen.
\end{document}