\documentclass[10pt,a4paper]{article}
\usepackage[utf8]{inputenc}
\usepackage[german]{babel}
\usepackage[T1]{fontenc}
\usepackage{amsmath}
\usepackage{amsfonts}
\usepackage{amssymb}
\usepackage{graphicx}
\begin{document}
\section{Clapeyron Gleichung}
Liefert Steigung aller Phasengrenzlinien im P-T Diagramm eines Reinstoffes:
\begin{equation}
\frac{dp}{dT}=\frac{\Delta S_m}{\Delta V_m}
\end{equation}
\begin{itemize}
\item fest $\leftrightarrow$ flüssig: siehe Schmelzpunkt
\item flüssig $\leftrightarrow$ gasförmig: 
\begin{equation}
\frac{dp}{dT}=\frac{\Delta H_{m,v}}{\Delta V_{m,v}T}\stackrel{ideale Gase}{\approx}\frac{\Delta H_{m,v}p}{RT^2}
\end{equation}
Dieser Fall wird auch als Clausius-Clapeyron Gleichung bezeichnet.
\item fest $\leftrightarrow$ gasförmig: 
\begin{equation}
\frac{ln(p)}{dT} \approx \frac{\Delta H_{m,sub}}{RT^2}
\end{equation}
\begin{itemize}
\item p = Druck
\item T = Temperatur
\item $\Delta S_m=$ Molare Entropieänderung
\item $\Delta V_m=$ Molare Volumenänderung
\item $\Delta H_{m,v}=$ Molare Verdampfungsenthalpie
\item $R=$ universelle Gaskonstante
\item $\Delta H_{m,sub}=$ Molare Subimationsenthalpie
\end{itemize}
\end{itemize}
\end{document}