\documentclass[10pt,a4paper]{article}
\usepackage[utf8]{inputenc}
\usepackage[german]{babel}
\usepackage[T1]{fontenc}
\usepackage{amsmath}
\usepackage{amsfonts}
\usepackage{amssymb}
\usepackage{graphicx}
\begin{document}
\section{Hauptsätze der Thermodynamik}
\begin{enumerate}
\setcounter{enumi}{-1}
\item Befinden sich zwei Körper jeder für sich im thermischen Gleichgewicht mit einem Probekörper, so sind sie auch untereinander im thermischen Gleichgewicht.
\item Innere Energie ist im abgeschlossenen System konstant. Am System verrichtete Arbeit, bzw. zugegebene Wärmemenge wird mit einem positiven und vom System verrichtete Arbeit bzw. abgegebene Wärmemenge mit einem negativen Vorzeichen versehen. Für die innere Energie U, die Wärme Q und die Arbeit W gilt:
\begin{equation}
\Delta U = \Delta Q + \Delta W
\end{equation}
Das Perpetuum Mobile 1. Art verstößt gegen den 1. Hauptsatz.
\item Entropiesatz: Wärme fließt von warm nach kalt und nicht umgekehrt. Die Gesamtentropie in einem isolierten System kann nie kleiner werden. Wenn die Entropie des Systems ihren Maximalwert hat befindet es sich im Gleichgewicht. \newline
Solange in einem abgeschlossenen System die Entropie gleich bleibt, sind alle Vorgänge, die sich in dem System abspielen reversibel. Sobald Entropie erzeugt wird ist das System irreversibel.
\begin{align}
\Delta S &\ge 0 \\
\Delta S &= \frac{\Delta Q_{rev}}{T} \\
\Delta S &= k_b ln(p) 
\end{align}
mit der Wahrscheinlichkeit p. \newline
Das Perpetuum Mobile 2. Art verstößt gegen den 2. Hauptsatz.
\item Es ist unmöglich die Temperatur eines Systems auf den absoluten Nullpunkt zu senken.
\end{enumerate}
\end{document}