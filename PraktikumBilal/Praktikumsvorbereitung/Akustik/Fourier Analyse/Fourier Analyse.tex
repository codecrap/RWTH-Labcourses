\documentclass[10pt,a4paper]{article}
\usepackage[utf8]{inputenc}
\usepackage[german]{babel}
\usepackage[T1]{fontenc}
\usepackage{amsmath}
\usepackage{amsfonts}
\usepackage{amssymb}
\usepackage{makeidx}
\usepackage{graphicx}
\author{Erik Zimmermann}
\begin{document}
\section{Fourier Reihe}
\begin{itemize}
\item diskret: 
\begin{equation}
 f(t)=A_0+\sum^{\infty}_{k=0}(A_k cos(\omega_kt)+B_k sin (\omega_kt))
\end{equation}
\end{itemize}
\hspace{1,0cm} mit $\frac{\omega_k=2\pi k}{T}$
\begin{align}
A_0&=\frac{1}{T}\int^{\frac{T}{2}}_{-\frac{T}{2}}f(t)dt\notag \\
A_k&=\frac{2}{T}\int^{\frac{T}{2}}_{-\frac{T}{2}}f(t)*cos(\omega_kt)dt\notag \\
B_k&=\frac{2}{T}\int^{\frac{T}{2}}_{-\frac{T}{2}}f(t)*sin(\omega_kt)dt\notag
\end{align}
$A_k=0$, falls $f(t)$ ungerade; $B_k=0$, falls $f(t)$ ungerade
\section{Fouriertransformation}
\begin{itemize}
\item Fouriertransformation zur Frequenzbestimmung:
\begin{equation}
f(\omega)=\frac{1}{\sqrt{2\pi}}\int^{\infty}_{-\infty}{e^{-it\omega}f(t)dt}\
\end{equation}
\item Rücktransformation:
\begin{equation}
f(t)=\frac{1}{\sqrt{2\pi}}\int^{\infty}_{-\infty}{e^{it\omega}f(\omega)dt}
\end{equation}
\item in Python mit FFT schnell lösbar, notfalls ablesen
\end{itemize}
\end{document}
