\documentclass[10pt,a4paper]{article}
\usepackage[utf8]{inputenc}
\usepackage[german]{babel}
\usepackage[T1]{fontenc}
\usepackage{amsmath}
\usepackage{amsfonts}
\usepackage{amssymb}
\usepackage{graphicx}
\begin{document}
\section{Doppler Effekt}
\begin{itemize}
\item Sender und Empfänger bewegen sich relativ zueinander.
\item Empfänger misst eine von der Senderfrequenz $f_0$ verschiedene Frequenz f.
\item Effekt ist nicht symmetrisch unter Vertauschung von Sender und Empfänger.
\item für die vom Empfänger registrierten Frequenzen gilt:
\begin{equation}
f=f_0\cdot \frac{1}{1\pm v/c} =
	\begin{cases}
	-: \text{Sender nähert sich} \\
	+: \text{Sender entfernt sich} 
	\end{cases}
\end{equation}
und 
\begin{equation}
f=f_0\cdot (1 \pm v/c) =
	\begin{cases}
	+: \text{Empfänger nähert sich} \\
	-: \text{Empfänger entfernt sich  sich}
	\end{cases}
\end{equation}
\end{itemize}
\subsection{Herleitung: Sender nähert sich}
Durch die Relativbewegung der Schallquelle zum Medium ändert sich für den Beobachter die Wellenlänge der Schallwelle.
\begin{align}
\lambda&=\lambda_0-vT \notag \\
\Rightarrow f&=\frac{c}{\lambda}=\frac{c}{\lambda_0-vT}=\frac{c}{\frac{c}{f_0}-vT}=f_0\cdot \frac{1}{1- v/c} \notag
\end{align}
\subsection{Herleitung: Empfänger nähert sich}
Wellenlänge ändert sich nicht, dafür aber die Relativgeschwindigkeit. Im ruhenden Fall ist $v_{rel}=c$. Hier gilt allerdings:
\begin{align}
v_{rel}=c+v \Rightarrow f=\frac{c+v}{\lambda_0}=\frac{c+v}{\frac{c}{f_0}}=f_0\cdot (1 + v/c) \notag
\end{align}
\end{document}